\documentclass{article}[11pt, lettersize]

\usepackage{amsthm}
\usepackage{enumitem}
\usepackage{amssymb}

\begin{document}

\author{Chris, Z'Nyah, Charles} 
\title{PUMP2: Problem Set 1}
\maketitle

\section{In-Set-tion}
\begin{enumerate}[label=(\alph*)]
	\item $ \{\varnothing \},
		\{1\},
		\{2\},
		\{cat\},
		\{dog\},
		\{1, 2\},
		\{1, cat\},
		\{1, dog\},
		\{2, cat\},
		\{2, dog\},
		\{cat ,dog\},
		\{1, 2, cat\},
		\{1, 2, dog\},
		\{1, cat, dog\},
		\{2, cat ,dog\},
		\{1, 2, cat, dog\}\\	
		\{\varnothing\},
		\{\varnothing, cat\},
		\{\varnothing, dog\},
		\{\varnothing, cat ,dog\},
		\{\{\}\},
		\{\{\}\},
		\{\{\}\},
		$
	\item $ \sum_{k=1}^{n} {n \choose k} = 2^n$
	\item 
	\begin{proof} 
		We check the base case. When the set $S$ has zero element, the powerset of $S$ has only one element, the null set which is $2^0=1$. As a result, the base case is true. As discovered above, we know that $ \sum_{k=1}^{n} {n \choose k} = 2^n$ We induct on n. We have to show that $ 2\sum_{k=1}^{n} {n \choose k} = (2^n+1)$. We see that $\sum_{k=1}^{n} {n \choose k} + \sum_{k=1}^{n} {n \choose k} = 2^n + 2^n = 2(2^n) = 2^(n+1)$ Thus the proposition stays true.  
	\end{proof}
\end{enumerate}
\end{document}
